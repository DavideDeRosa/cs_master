\documentclass[12pt]{article}
\usepackage{graphicx}
\usepackage{cite}
\usepackage{geometry}

\geometry{margin=1in}

\begin{document}
\begin{center}
\LARGE
Apache Openwhisk \vspace{10pt}\\
\Large
Distributed Software Systems, 2024/25 \vspace{10pt}\\
\large
Davide De Rosa
\end{center}

\section{Description of Apache Openwhisk}
What it is: Briefly introduce Apache OpenWhisk as an open-source, serverless cloud platform designed to execute functions in response to events.\\
Core concept: Describe its Function-as-a-Service (FaaS) nature and event-driven approach.\\
Key features: Outline the system’s capabilities, such as dynamic scaling, event-triggered function execution, and integration with other cloud services and APIs.

\section{Context}
Main use cases: List some typical scenarios where OpenWhisk is commonly applied:\\
Event-driven processing, such as responding to webhooks or changes in a database.\\
Real-time data processing, like stream processing from IoT devices.\\
Serverless backend for applications, where each API endpoint triggers a different function.\\
Context Diagram: Include a high-level context diagram showing interactions between OpenWhisk, event sources (like APIs, databases, or user actions), and other services.\\
Example scenario: Describe a scenario, e.g., a photo-sharing app where OpenWhisk triggers a function to generate thumbnails whenever a user uploads an image.\\
History and Invention: Explain that OpenWhisk was created by IBM to fill the need for scalable, event-driven computing on the cloud, with contributions from the open-source community leading to its incubation and adoption by the Apache Software Foundation.

\section{Main Architectural Drivers}
Scalability: Explain that OpenWhisk is built to handle variable loads, adjusting function instances as needed without requiring pre-allocated server resources.\\
Low Latency: Address OpenWhisk’s design for quick response times, especially in real-time processing scenarios.\\
Fault Tolerance: Describe mechanisms for error handling and retry policies, which ensure robustness in the face of transient failures.\\
Vendor-Neutral: Highlight OpenWhisk’s open-source nature, allowing it to run on multiple cloud platforms or on-premises infrastructure, making it vendor-neutral.\\
Flexibility in Integration: Describe how OpenWhisk’s modular design allows it to connect with various services, enabling it to support diverse event sources and destinations.

\section{Structure}
Patterns: Discuss patterns OpenWhisk uses, like microservices and event-driven processing.\\
Components and Connectors:\\
Controller: Responsible for routing requests, managing APIs, and authentication.\\
Invoker: Executes functions in response to triggers, managing container lifecycles.\\
Messaging System: Connects event sources to functions.\\
Database: Stores metadata about functions, triggers, and actions (often uses CouchDB or similar).\\
Connector components: Enable integration with third-party services and external event sources.

\section{Behavior}
How OpenWhisk works:\\
Trigger-Action Model: Describe the core event-driven behavior, where functions (actions) are invoked by specific triggers.\\
Execution Flow: Explain the flow when an event occurs:\\
An event source triggers an action.\\
The Controller routes the request to an Invoker.\\
The Invoker pulls an image from a container registry and runs the function.\\
The system processes the output and sends it back to the client or next service.\\
Concurrency and Isolation: Describe how OpenWhisk manages concurrency using containers to isolate function executions.

\section{Rationale}
Why OpenWhisk has this architecture: Explain the design choices for the following reasons:\\
Containerized functions: Containers are lightweight and isolated, providing a scalable and secure way to handle function executions.\\
Event-driven model: Emphasizes resource efficiency and aligns with many real-time and sporadic-use cases common in serverless applications.\\
Microservices Architecture: Allows modular, flexible, and resilient system design, making it easier to add or modify individual components.

\section{Similar or Competing Systems/Middleware}
AWS Lambda: Compare OpenWhisk with Lambda, discussing differences like deployment options (AWS-specific vs. multi-cloud) and the ecosystem of integrated services.\\
Google Cloud Functions: Highlight differences in implementation, integration with Google’s ecosystem, and scaling approaches.\\
Microsoft Azure Functions: Describe how Azure Functions differs in pricing, integration, and scaling compared to OpenWhisk.\\
Kubeless or Knative: Mention these Kubernetes-based serverless platforms as alternatives, especially for organizations invested in Kubernetes.

\section{Simple Implementation Experiment}
Experiment Idea: Outline a basic OpenWhisk implementation on a cloud provider (IBM Cloud or a local installation with Kubernetes). A simple experiment might include:\\
Setting up OpenWhisk and creating a basic function, like a “Hello World” HTTP endpoint.\\
Configuring a trigger, e.g., making the function respond to HTTP requests.\\
Deploying and testing the function to ensure it executes correctly.\\
Reporting Results: Document any performance measurements, such as response time and scalability observations, or any challenges faced during deployment.

\begin{thebibliography}{9}
    \bibitem{sample1}
    Author, A. (Year). \textit{Title of the paper}. Journal Name, Volume(Issue), pages.
\end{thebibliography}

\end{document}