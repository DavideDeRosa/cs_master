\documentclass{article}

\usepackage[utf8]{inputenc}
\usepackage[T1]{fontenc}
\usepackage{lipsum}
\usepackage{graphicx}
\usepackage{amsmath}
\usepackage[margin=1in]{geometry}
\usepackage{titlesec}
\usepackage{parskip}
\usepackage{tcolorbox}

\titleformat{\section}
{\LARGE\bfseries}{\thesection}{1em}{}

\titleformat{\subsection}
{\Large\bfseries}{\thesection}{1em}{}

\begin{document}

\pagestyle{empty}

\section*{Algebraic Effects}
\large

Gli Algebraic Effects si basano su un concetto fondamentale: l'istruzione \texttt{throw(E)} trasferisce il controllo a distanza alla clausola \texttt{catch} che gestisce l'effetto \texttt{E}. Il codice remoto può successivamente restituire il controllo alla \texttt{throw} fornendo un valore \texttt{V} che diventa il risultato dell'istruzione \texttt{throw} stessa.

Un esempio in pseudo codice Erlang potrebbe essere:
\begin{tcolorbox}
\begin{verbatim}
2 * try 3 + (5 + throw 7)
    catch
        0 -> 22                     % come un'eccezione normale
        N -> resume N+1             % resume nuova keyword
    end
\end{verbatim}
\end{tcolorbox}
Nel caso dell'utilizzo di resume, il risultato sarebbe \texttt{32}.

Come viene implementato questo meccanismo di "ritorno"?\\
Nelle eccezioni tradizionali, la \texttt{throw} esegue un ciclo \texttt{while} sullo stack e per ogni record di attivazione/record try catch che non gestisce l'eccezione effettua \texttt{Stack.pop()}.

Negli algebraic effects (nell'implementazione più semplice) la \texttt{throw}, invece di eseguire un semplice \texttt{Stack.pop()}, effettua:
\begin{tcolorbox}
\begin{verbatim}
RA = Stack.pop();
Detached.push(RA);
\end{verbatim}
\end{tcolorbox}
dove Detached rappresenta un secondo stack, ordinato in senso inverso, nel quale vengono temporaneamente memorizzati i frame di stack distaccati.

L'istruzione \texttt{resume} esegue la seguente operazione:
\begin{tcolorbox}
\begin{verbatim}
while(!Detach.is_empty()) {
    RA = Detach.pop();
    Stack.push(RA);
}
\end{verbatim}
\end{tcolorbox}
assegnando il risultato della \texttt{resume} come valore di ritorno della \texttt{throw}.

Se invece nel ramo \texttt{catch} non viene utilizzata l'istruzione \texttt{resume}:
\begin{tcolorbox}
\begin{verbatim}
while(!Detach.is_empty())
    Detach.pop()
\end{verbatim}
\end{tcolorbox}
\vspace{8pt}
In un linguaggio con supporto per gli effetti algebrici viene introdotto un nuovo tipo di dato astratto denominato \textbf{fibra} (\textbf{fiber}) che rappresenta un frammento di stack distaccato. Questo è un tipo di dato di prima classe sul quale è possibile invocare l'operazione \texttt{resume()} per reinstallarlo in cima allo stack corrente. Il ramo \texttt{catch} cattura questa fibra.

Esaminiamo ora un esempio di implementazione di uno scheduler, in pseudo codice Erlang:
\begin{tcolorbox}
\begin{verbatim}
yield() -> throw(yield).            % yield è un effetto
fork(G) -> throw(fork).             % fork è un effetto

code_to_fiber(F) ->
 try
  throw(stop),F
 catch
  stop, K -> K
 end.

% Main è il primo thead da eseguire
% Queue la coda dei thread sospesi
scheduler(Main, Queue) ->
 try
  resume(Main, ok)
 catch
  yield, K ->                      % K è la fibra, i pattern sono sempre
   case Queue of                   % pattern_eccezione + pattern K
    [] -> scheduler(K, [])
    [F|L] -> scheduler(F, append(L,[K]))
   end ;
  fork(G), K ->
    scheduler(K, append(Queue,[code_to_fiber(G)]))
 end.

scheduler(Main) ->
 scheduler(code_to_fiber(Main),[]).
\end{verbatim}
\end{tcolorbox}
\vspace{8pt}
Gli effetti algebrici consentono quindi di gestire in maniera non locale gli errori come se fossero gestiti localmente, di implementare scheduler a livello utente (user-space), e altri meccanismi avanzati.

Un'implementazione efficiente degli effetti algebrici trasforma lo stack in uno stack di fibre, dove ogni fibra è a sua volta uno stack.\\
In questo modo, le operazioni di distacco e riattacco di una fibra hanno un costo computazionale di \textbf{O(1)}, sebbene ciò comporti che lo stack non sia più allocato in modo contiguo in memoria.

\end{document}