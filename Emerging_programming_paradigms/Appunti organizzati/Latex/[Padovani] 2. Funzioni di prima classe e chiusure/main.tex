\documentclass{article}

\usepackage[utf8]{inputenc}
\usepackage[T1]{fontenc}
\usepackage{lipsum}
\usepackage{graphicx}
\usepackage{amsmath}
\usepackage[margin=1in]{geometry}
\usepackage{titlesec}
\usepackage{parskip}
\usepackage{tcolorbox}

\titleformat{\section}
{\LARGE\bfseries}{\thesection}{1em}{}

\titleformat{\subsection}
{\Large\bfseries}{\thesection}{1em}{}

\begin{document}

\pagestyle{empty}

\section*{Funzioni di prima classe e chiusure}
\large

In un linguaggio funzionale tendenzialmente le funzioni sono \textbf{entità di prima classe}.\\
Un'entità di prima classe può essere:
\begin{itemize}
    \item \textbf{Argomento} di una funzione.
    \item \textbf{Risultato} di una funzione.
    \item \textbf{Definita} dentro una funzione.
    \item \textbf{Assegnata} a una variabile.
    \item \textbf{Memorizzata} in una struttura dati (array, liste, alberi, \dots).
\end{itemize}
In generale, un'entità di prima classe in un \textit{linguaggio tipato} ha un \textbf{tipo che la descrive}, andando a descrivere il tipo del dominio (\textit{input}) e del codominio (\textit{output}).

Vengono ora osservati diversi linguaggi, per determinare se sono linguaggi funzionali o meno:
\begin{itemize}
    \item \textbf{C}: il linguaggio C \textbf{non} è un linguaggio funzionale.
    
    In C esistono dei \textbf{puntatori a funzioni}, ma non è possibile definire funzioni dentro funzioni.
    \item \textbf{Pascal}: anche il linguaggio Pascal \textbf{non} è un linguaggio funzionale.
    
    In Pascal è possibile definire \textbf{funzioni dentro funzioni}, ma non c'è un tipo \texttt{funzione}.
    \item \textbf{Haskell}: \textbf{è} un linguaggio funzionale.
    
    Le funzioni hanno un tipo, possono essere definite dentro ad altre funzioni e possono essere risultati di altre funzioni.
    \item \textbf{Java, Python, C++, \dots}: non sono pienamente linguaggi funzionali. 
    
    Le funzioni vengono modellate come oggetti (e incorporate a posteriori). Viene applicato dello zucchero sintattico per le $\lambda$ espressioni (ad esempio, in Java: \texttt{(a, b) -> ...;}). Inoltre, vengono utilizzate delle interfacce funzionali, come il \texttt{Comparator} in Java.
\end{itemize}
\vspace{8pt}
Un semplice test per comprendere se all'interno di un linguaggio le funzioni sono entità di prima classe è il \textbf{definire la composizione funzionale $(f \circ g)(x) = f(g(x))$}.

Viene scelto questo test dato che la composizione funzionale prende in input due funzioni e ne restituisce una in output.

Vediamo ora due esempi di implementazione.\\
In Haskell, preso dalla libreria standard:
\begin{tcolorbox}
\begin{verbatim}
(.) :: (b -> c) -> (a -> b) -> a -> c
(.) f g = \x -> f (g x)
\end{verbatim}
\end{tcolorbox}

Nota: \texttt{f (g x)} è equivalente al solito \texttt{f(g(x))}. Viene scelta questa notazione per semplificare l'utilizzo del linguaggio, evitando l'uso eccessivo di parentesi.\\
Inoltre, la notazione \texttt{$\backslash$x} indica una nuova funzione che si aspetta in input l'argomento \texttt{x}.\\
Il \texttt{$\backslash$} vuole richiamare ad una $\lambda$ stilizzata.

In Java:
\begin{tcolorbox}
\begin{verbatim}
public static <A,B,C>
Function<A,C> compose(Function<B,C> f, Function<A,B> g) {
    return x -> f.apply(g.apply(x));
}
\end{verbatim}
\end{tcolorbox}

Il metodo si aspetta due funzioni che andranno a produrre una funzione come output. Tutte queste funzioni sono in realtà oggetti che implementano una determinata interfaccia (\texttt{Function}).\vspace{14pt}\\
Proviamo invece ora ad implementarlo in C:
\begin{tcolorbox}
\begin{verbatim}
int (*)(int) compose(int (*f)(int), int (*g)(int)) {
    int aux(int x) {
        return f(g(x));
    }
    return aux;
}

int main() {
    int (*plus2)(int) = compose(succ, succ);
    printf("%d", plus2(1));
}
\end{verbatim}
\end{tcolorbox}

Ci sono però dei problemi. Stiamo richiamando valori al di fuori della funzione \texttt{aux} (\texttt{f} e \texttt{g} che appartengono a \texttt{compose}).

Quando viene eseguito il corpo di \texttt{aux}, gli slot che contengono i valori di \texttt{f} e \texttt{g} non esistono più. Una funzione è una \textit{computazione ritardata} che può accedere a nomi definiti all’esterno del suo corpo, ma che potrebbero non esistere più nel momento in cui il corpo viene eseguito.\vspace{14pt}\\
Proviamo ora ad implementarlo anche in Pascal:
\begin{tcolorbox}
\begin{verbatim}
function E(x: real): real;
    function F(y: real): real;
    begin
        F := x + y
    end;
begin
    E := F(3) + F(4)
end;
\end{verbatim}
\end{tcolorbox}

In C le funzioni non possono essere annidate, in Pascal si. Se una funzione accede a nomi definiti all’esterno del suo corpo, deve trattarsi di variabili \textbf{globali} (o di altre funzioni \textbf{globali}).
Variabili/funzioni globali esistono per tutta la durata del programma, quindi siamo salvi.

In Pascal le funzioni possono essere annidate, ma non restituite come risultato. Se una funzione accede a nomi definiti all’esterno del suo corpo, deve trattarsi di variabili \textbf{globali} o di \textbf{variabili locali ancora esistenti}.

Viene quindi utilizzato un meccanismo chiamato \textbf{puntatore di catena statica}. Il puntatore di catena statica è il collegamento che ogni activation record (di una funzione annidata) mantiene con il proprio ambiente di definizione. Questo collegamento è fondamentale per permettere alla funzione annidata di accedere correttamente alle variabili definite nei livelli esterni. È una soluzione elegante per gestire la visibilità e la durata delle variabili in presenza di funzioni annidate, garantendo che le variabili locali di un contesto esterno siano ancora accessibili fintanto che ne esiste un riferimento valido.

\subsection*{Chiusure}
In un linguaggio funzionale una funzione può essere invocata in un luogo e in un tempo molto diversi da quelli in cui la funzione è stata definita.

Per assicurare che l’esecuzione del corpo della funzione proceda senza intoppi (ovvero che tutti i dati di cui ha bisogno esistono ancora) tali dati vengono copiati e “impacchettati” insieme al codice della funzione in una cosiddetta \textbf{chiusura}.
\begin{center}
    \textbf{Chiusura} = \textbf{codice della funzione} + \textbf{(valori delle) variabili libere}.    
\end{center}
\vspace{14pt}

Le chiusure possono avvere molti utilizzi. Un possibile caso d'uso è il \textbf{currying}.

In molti linguaggi funzionali, le funzioni “a più argomenti” sono “cascate” di funzioni a un singolo argomento. Due possibili esempi possono essere:
\begin{tcolorbox}
\begin{verbatim}
add :: Int -> Int -> Int
add = \x -> \y -> x + y
\end{verbatim}
\end{tcolorbox}
\begin{tcolorbox}
\begin{verbatim}
leq :: Int -> Int -> Bool
leq = \x -> \y -> x <= y
\end{verbatim}
\end{tcolorbox}

Questo consente di non avere una nozione nativa di "funzione a più argomenti", rendendo quindi il linguaggio più semplice.

Inoltre, le funzioni "currificate" possono essere \textbf{specializzate}. Un possibile esempio dove si specializza una funzione è il seguente, riprendendo la funzione \texttt{add} definita precedentemente:
\begin{tcolorbox}
\begin{verbatim}
>let f = add 1
>f 5
6
>f 7
8
\end{verbatim}
\end{tcolorbox}
Si può notare quindi come la funzione \texttt{add} sia stata specializzata come una funzione che somma \texttt{1} al numero passato come argomento. 

\pagebreak

Un ulteriore utilizzo delle chiusure è quello di \textbf{incapsulare} un dato sensibile creato localmente da un'altra funzione.


Un possibile esempio di quest'applicazione può essere, in linguaggio OCaml:
\begin{tcolorbox}
\begin{verbatim}
let make_counter () =
    let c = ref 0 in
    (fun () -> !c),
    (fun () -> c := !c + 1),
    (fun () -> c := 0)

let get, inc, reset = make_counter ()
\end{verbatim}
\end{tcolorbox}

In questo caso, \texttt{make\_counter} crea un riferimento \texttt{c} e tre funzioni per leggere, incrementare e resettare il contenuto di \texttt{c}.

Il contatore è incapsulato nella chiusura di (ed utilizzabile solo attraverso) queste funzioni.\vspace{14pt}\\
Altro caso d'uso comune per le chiusure è il loro utilizzo per \textbf{impacchettare} espressioni che non devono essere valutate subito, ma solo se necessario.

Questo concetto è allineato con i linguaggi \textbf{lazy}, come Haskell, dove gli argomenti di una funzione vengono valutati al massimo una volta e solo se necessario.

Un possibile esempio di quest'applicazione può essere:
\begin{tcolorbox}
\begin{verbatim}
zif f []       _        = []
zif f _        []       = []
zif f (x : xs) (y : ys) = f x y : zif f xs ys

sorted xs = and (zif (<=) xs (tail xs))
\end{verbatim}
\end{tcolorbox}

La funzione \texttt{zif} permette di controllare se una lista è ordinata. Un modo elegante di effettuare questo controllo è quello di confrontare le coppie che si ottengono unendo un valore con il suo successivo all'interno della lista.

\texttt{(zif (<=) xs (tail xs))} permette quindi di confrontare la lista con la sua coda (la lista senza il primo elemento).

\texttt{(tail xs)} potrebbe però provocare dei problemi, non essendo definita in caso di \texttt{xs} vuota.

Essendo però Haskell un linguaggio lazy, \texttt{sorted} è corretta perché quando \texttt{xs} è vuota il secondo argomento di \texttt{zif} non viene valutato.

Questa logica in Haskell viene anche applicata in contesti di base, come l'implementazione degli operatori logici (come l'operatore \texttt{AND \&\&}) e dell'if-else. Non vengono quindi valutati gli argomenti non necessari.

\pagebreak

\subsection*{Chiusure in Java}
Altri linguaggi moderni hanno implementato costrutti della programmazione funzionale all'interno del proprio linguaggio. Un esempio sono le chiusure implementate in Java.

Un possibile esempio di chiusura in Java può essere:
\begin{tcolorbox}
\begin{verbatim}
public static Function<Integer,Integer> add(int x) {
    return y -> x + y;
}
\end{verbatim}
\end{tcolorbox}

In questo caso, la variabile \texttt{x} è \textbf{effectively final}. Il valore di \texttt{x} non deve essere alterato all'interno del codice della chiusura.

La chiusura contiene \textbf{copie} dei valori delle variabili libere. Se tali variabili sono modificate, la semantica della chiusura può risultare oscura.\\
Se \texttt{x} fosse un riferimento ad un oggetto la situazione si andrebbe a complicare, potendo modificare lo stato interno dell'oggetto successivamente.

In un linguaggio funzionale \textbf{puro} il problema non si pone perché non c’è assegnamento.\vspace{14pt}\\
Volendo quindi confrontare \textit{oggetti} e \textit{chiusure}:
\begin{itemize}
    \item All'interno di oggetti sono presenti \textbf{(campi + riferimenti a metodi)}, mentre nelle chiusure sono presenti \textbf{(valori della variabili libere + riferimento a funzione)}.
    \item Negli oggetti i \textbf{campi} sono \textbf{mutabili}, mentre nelle chiusure i \textbf{valori} sono \textbf{immutabili}.
    \item Negli oggetti i metodi hanno un parametro implicito (tipicamente \texttt{self} o \texttt{this}) con riferimento all'oggetto ricevente, mentre nelle chiusure la funzione ha un parametro implicito con riferimento alla chiusura in cui la funzione trova i valori delle variabili libere.
\end{itemize}

\end{document}